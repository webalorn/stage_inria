\documentclass{article}

\usepackage[francais]{babel}
\def\printlandscape{\special{landscape}}    % Works with dvips.
%\usepackage{pstricks,pst-node,pst-tree}
%\usepackage{amssymb}
\usepackage[utf8]{inputenc}
\usepackage[T1]{fontenc} 
\usepackage{fancybox} % for shadow and Bitemize
\usepackage{alltt}
\usepackage{graphicx}
%\usepackage{epsfig}
%\usepackage{fullpage}
%\usepackage{fancyhdr}
%\usepackage{moreverb}
%\usepackage{xspace}
\usepackage[colorlinks,hyperindex,bookmarks,linkcolor=blue,citecolor=blue,urlcolor=blue]{hyperref}

\usepackage{wrapfig}
\usepackage{epsf}

\title{Rapport de TER}
\author{Martin Strecker\\
\url{http://www.irit.fr/~Martin.Strecker}}
\date{\today}
         
\begin{document}

\maketitle
\tableofcontents

\begin{abstract}
Résumé du contenu du document.
\end{abstract}

%-----------------------------------------------------------
\section{Introduction}\label{sec:intro}

Prérequis:
\begin{itemize}
\item Un fichier \texttt{*.tex}, par exemple \texttt{rapport.tex}, qui
  contient le texte.
\item Un fichier \texttt{*.bib}, par exemple \texttt{rapport.bib}, qui
  contient les références bibliographiques.
\end{itemize}

Utilisation de Latex:
\begin{itemize}
\item Lancer \texttt{pdflatex rapport} pour compilation du fichier
  \texttt{rapport.tex}
\item Lancer \texttt{biblatex rapport} pour compilation du fichier
  \texttt{rapport.bib}
\item Répéter \texttt{pdflatex rapport} deux fois pour prendre en
  compte des modifications
\end{itemize}

Documentation: 
\begin{itemize}
\item \LaTeX: \url{http://www.latex-project.org/} 
\item \TeX users group: \url{http://www.tug.org/}
\end{itemize}


%-----------------------------------------------------------
\section{Présentation du projet}

\subsection{But du projet}
On a utilisé
\cite{lindholm99_java_virtual_machin_specif,moore89_system_verif} et
aussi \cite{strecker02_verif_java_compil}.

\subsection{Implantation}

\begin{figure}[htbp]
  \centering
  \includegraphics[scale=0.5]{w_gandalf5.jpg}
  \caption{Une petite image}
  \label{fig:im}
\end{figure}

Inclusion d'images en format PDF (création par exemple avec
XFig\footnote{Page web: \url{http://www.xfig.org/}})

\subsection{Spécialités}

Facile à utiliser: Listes d'éléments numérotés:
\begin{enumerate}
\item premier élément
\item deuxième élément
\item aussi imbriqués:
%
\begin{itemize}
\item Ici, une liste non numérotée
\item avec un autre élément
\end{itemize}
%
\end{enumerate}

Mathématiques:
\begin{itemize}
\item Sous- et super-scripts: $a_n$ et $b^k$, utiliser des accolades
  pour des expressions plus complexes: $x^{(y^z)}$

\item Symboles mathématiques, tels que $\sum_{i=0}^{n} f(i)$

\item Caractères de l'alphabet grecque: $\Gamma$ ou calligraphiques: ${\cal D}$
\end{itemize}

Fontes de caractères spécifiques:
\begin{itemize}
\item \emph{Italique}
\item \textbf{Gras}
\item \texttt{typewriter}
\item ou combiné \emph{\textbf{italique-gras}}
\end{itemize}

%-----------------------------------------------------------
\section{Gestion du projet}

\begin{tabular}[h]{|l|l|}
\hline
Dates              & Tâches  \\
\hline
\hline
Du 7 mars au 2 mai & 1~ière itération \\
                   & 2~ième  itération \\
\hline
\end{tabular}

Je me réfère à la section~\ref{sec:intro} et la figure~\ref{fig:im}.

%-----------------------------------------------------------
\bibliography{rapport}
\bibliographystyle{abbrv}
\end{document}

%%% Local Variables:
%%% mode: latex
%%% TeX-master: t
%%% coding: utf-8
%%% End:
